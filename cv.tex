%-------------------------
% Resume in Latex
% Author : Jake Gutierrez
% Based off of: https://github.com/sb2nov/resume
% License : MIT
%------------------------

\documentclass[a4,12pt]{article}

\usepackage{latexsym}
\usepackage[empty]{fullpage}
\usepackage{titlesec}
\usepackage{marvosym}
\usepackage[usenames,dvipsnames]{color}
\usepackage{verbatim}
\usepackage{enumitem}
\usepackage[hidelinks]{hyperref}
\usepackage{fancyhdr}
\usepackage[english]{babel}
\usepackage{tabularx}
\usepackage{fontawesome}
\usepackage{multicol}
\setlength{\multicolsep}{-3.0pt}
\setlength{\columnsep}{-1pt}
\input{glyphtounicode}

\usepackage[super]{nth}


%----------FONT OPTIONS----------
% sans-serif
% \usepackage[sfdefault]{FiraSans}
% \usepackage[sfdefault]{roboto}
% \usepackage[sfdefault]{noto-sans}
% \usepackage[default]{sourcesanspro}

% serif
% \usepackage{CormorantGaramond}
% \usepackage{charter}


\pagestyle{fancy}
\fancyhf{} % clear all header and footer fields
\fancyfoot{}
\renewcommand{\headrulewidth}{0pt}
\renewcommand{\footrulewidth}{0pt}

% Adjust margins
\addtolength{\oddsidemargin}{-0.6in}
\addtolength{\evensidemargin}{-0.5in}
\addtolength{\textwidth}{1.19in}
\addtolength{\topmargin}{-.7in}
\addtolength{\textheight}{1.4in}

\urlstyle{same}

\raggedbottom
\raggedright
\setlength{\tabcolsep}{0in}

% Sections formatting
\titleformat{\section}{
  \vspace{-4pt}\scshape\raggedright\large\bfseries
}{}{0em}{}[\color{black}\titlerule \vspace{-5pt}]

% Ensure that generate pdf is machine readable/ATS parsable
\pdfgentounicode=1

%-------------------------
% Custom commands
\newcommand{\resumeItem}[1]{
  \item\small{
    {#1 \vspace{-2pt}}
  }
}

\newcommand{\classesList}[4]{
    \item\small{
        {#1 #2 #3 #4 \vspace{-2pt}}
  }
}

\newcommand{\resumeSubheading}[4]{
  \vspace{-2pt}\item
    \begin{tabular*}{1.0\textwidth}[t]{l@{\extracolsep{\fill}}r}
      \textbf{#1} & \textbf{\small #2} \\
      \textit{\small#3} & \textbf{\small #4} \\
    \end{tabular*}\vspace{-7pt}
}

\newcommand{\resumeSubheadingWithoutBottomLine}[2] {
    \vspace{-2pt}\item
    \begin{tabular*}{1.0\textwidth}[t]{l@{\extracolsep{\fill}}r}
      \textbf{#1} & \textbf{\small #2} \\
    \end{tabular*}\vspace{-7pt}
}

\newcommand{\resumeSubSubheading}[2]{
    \item
    \begin{tabular*}{0.97\textwidth}{l@{\extracolsep{\fill}}r}
      \textit{\small#1} & \textit{\small #2} \\
    \end{tabular*}\vspace{-7pt}
}

\newcommand{\resumeProjectHeading}[2]{
    \item
    \begin{tabular*}{1.001\textwidth}{l@{\extracolsep{\fill}}r}
      \small#1 & \textbf{\small #2}\\
    \end{tabular*}\vspace{-7pt}
}

\newcommand{\resumeSubItem}[1]{\resumeItem{#1}\vspace{-4pt}}

\renewcommand\labelitemi{$\vcenter{\hbox{\tiny$\bullet$}}$}
\renewcommand\labelitemii{$\vcenter{\hbox{\tiny$\bullet$}}$}

\newcommand{\resumeSubHeadingListStart}{\begin{itemize}[leftmargin=0.0in, label={}]}
\newcommand{\resumeSubHeadingListEnd}{\end{itemize}}
\newcommand{\resumeItemListStart}{\begin{itemize}}
\newcommand{\resumeItemListEnd}{\end{itemize}\vspace{-5pt}}

%-------------------------------------------
%%%%%%  RESUME STARTS HERE  %%%%%%%%%%%%%%%%%%%%%%%%%%%%


\begin{document}

%----------HEADING----------

\begin{center}
    {\Huge \scshape Denis Shabanov} \\ \vspace{1pt}
    College Ring 7 (XC-341), 28759 Bremen, Germany \\ \vspace{1pt}
    \small \raisebox{-0.1\height}\faPhone\
    \href{tel:+4901748549668}{\underline{+49 (0174) 854-96-68}} ~
    \href{
    mailto:shadenis2015@gmail.com
    }{\raisebox{-0.2\height}\faEnvelope\  \underline{
    shadenis2015@gmail.com
    }} ~ 
    \href{https://t.me/mmel0n}{\raisebox{-0.2\height}\faSend\ \underline{@mmel0n}}  ~
    \href{https://github.com/dodolfin}{\raisebox{-0.2\height}\faGithub\ \underline{github.com/dodolfin}}
    \vspace{-8pt}
\end{center}


%-----------EDUCATION-----------
\section{Education}
  \resumeSubHeadingListStart
    \resumeSubheading
      {Saint Petersburg State University}{09/2021 — 06/2022}
      {Applied Mathematics and Computer Science B.Sc.}{GPA: 3.603}
    \resumeSubheading
      {Constructor University (formerly Jacobs), Bremen}{09/2022 — Expected 05/2024}
      {Computer Science B.Sc.}{\textbf{GPA: Not yet applicable}}
  \resumeSubHeadingListEnd
  

%-----------PROGRAMMING SKILLS-----------
\section{Skills}
 \begin{itemize}[leftmargin=0.15in, label={}]
    \small{\item{
     \textbf{Languages}{: C++, Kotlin, Java, Python} \\
     \textbf{Technologies}{: Linux, Git, Flask, PostgreSQL} \\
     \textbf{Relevant Coursework}{: Algorithms \& Data~Structures, Object-Oriented Programming, Databases \& Web~Services, Big Data Software Engineering (in progress, distributed KV storage)}
    }}
 \end{itemize}
 \vspace{-16pt}


% ------RELEVANT COURSEWORK-------
% \section{Relevant Coursework}
%     %\resumeSubHeadingListStart
%         \begin{multicols}{4}
%             \begin{itemize}[itemsep=-5pt, parsep=3pt]
%                 \item\small Data Structures
%                 \item Software Methodology
%                 \item Algorithms Analysis
%                 \item Database Management
%                 \item Artificial Intelligence
%                 \item Internet Technology
%                 \item Systems Programming
%                 \item Computer Architecture
%             \end{itemize}
%         \end{multicols}
%         \vspace*{2.0\multicolsep}
%     %\resumeSubHeadingListEnd


%-----------EXPERIENCE-----------
% \section{Experience}
%   \resumeSubHeadingListStart

%     \resumeSubheading
%       {„Sirius“ Educational Centre}{December 2019}
%       {}{Sochi, Russia}
%       \resumeItemListStart
%         \resumeItem{Developed a service to automatically perform a set of unit tests daily on a product in development in order to decrease time needed for team members to identify and fix bugs/issues.}
%         \resumeItem{Incorporated scripts using Python and PowerShell to aggregate XML test results into an organized format and to load the latest build code onto the hardware, so that daily testing can be performed.}
%         \resumeItem{Utilized Jenkins to provide a continuous integration service in order to automate the entire process of loading the latest build code and test files, running the tests, and generating a report of the results once per day.}
%         \resumeItem{Explored ways to visualize and send a daily report of test results to team members  using HTML, Javascript, and CSS.}
%       \resumeItemListEnd
    
%   \resumeSubHeadingListEnd
% \vspace{-16pt}

%-----------PROJECTS-----------
\section{Projects}
    \vspace{-5pt}
    \resumeSubHeadingListStart
          \resumeProjectHeading
          {\href{https://github.com/dodolfin/spbu-viz}{\textbf{Visualization Tool}} $|$ \emph{Kotlin, Apache Batik, Clikt}}{October 2021}
          \resumeItemListStart
            \resumeItem{A tool that renders SVG charts using data from external CSV file.}
            \resumeItem{I used Apache Batik library to work with SVGs as with java.awt.Graphics instances.}
          \resumeItemListEnd
          \vspace{-13pt}
          
        \resumeProjectHeading
          {\href{https://github.com/dodolfin/spbu-kvdb}{\textbf{Key-Word Database}} $|$ \emph{Kotlin, kotlinx-serialization, Clikt}}{October 2021}
          \resumeItemListStart
            \resumeItem{A dbm-inspired key-word database management tool.}
            \resumeItem{Used kotlinx-serialization to read and write from JSON file.}
            \resumeItem{Clikt was used to design command-line interface.}
          \resumeItemListEnd
          \vspace{-13pt}


      \resumeProjectHeading
          {\href{https://github.com/dodolfin/spbu-diff}{\textbf{Diff Tool}} $|$ \emph{Kotlin}}{September 2021}
          \resumeItemListStart
            \resumeItem{A simplified clone of an UNIX diff utility.}
            \resumeItem{Uses naive $O(n^2)$ longest common subsequence finding algorithm.}
            \resumeItem{Produces output in normal (Linux-like) and unified (GitHub-like) modes.}
          \resumeItemListEnd

      \resumeProjectHeading
          {\href{https://github.com/dodolfin/c-cpp-homework}{\textbf{C, C++ Homework}} $|$ \emph{C, C++}}{November 2021 — May 2021}
          \resumeItemListStart
            \resumeItem{\href{https://drive.google.com/file/d/1ib23DhRDuYS8YUAsHL0pgDelphTkBTYT/view?usp=sharing}{\underline{.zip repo (in Russian)} } }
            \resumeItem{C homeworks: Makefile, implementation of some functions from cstring, mergesort implementation, intrusive list implementation. C++ homeworks: std::vector implementation, std::shared\_ptr, C++ I/O.}
            \resumeItem{Most complex assignment in C: console utility that crops a rectangular segment of a BMP image and rotates it 90 degrees.}
            \resumeItem{Most complex assignment in C++: console utility that implements Huffman encoding and allows to compress all types of files.}
          \resumeItemListEnd

          
    \resumeSubHeadingListEnd
\vspace{-15pt}




%-----------ACHIEVEMENTS---------------
\section{Achievements}
    \resumeSubHeadingListStart
        \resumeSubheadingWithoutBottomLine{Participation in Russian Competitive Programming Contests}{2019 -- Spring 2021}
            \resumeItemListStart
%                \resumeItem{Majority of Russian “ICT” contests are actually programming contests with algorithmic problems (e.g., Codeforces problems)}
                \resumeItem{\textbf{\href{https://neerc.ifmo.ru/school/ioip/standings-2021.html}{High School Student Olympiad in Informatics and Programming (2021)}}: 43rd place out of 378 participants }
                \resumeItem{\textbf{\href{https://sesc.nsu.ru/upload/iblock/eb6/2019_3_inf_r.pdf}{All-Siberian Olympiad in Informatics (2020)}}: 77th place out of 540 participants}
                \resumeItem{\textbf{\href{http://web.archive.org/web/20200929173737/https://olympiada.spbu.ru/wp-content/uploads/info_zakl.pdf}{Saint Petersburg State University Olympiad in Informatics (2020)}}: 17th place out of more than 300 participants}

            \resumeItemListEnd
        
    \resumeSubHeadingListEnd


\end{document}

